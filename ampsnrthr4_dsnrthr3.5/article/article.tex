\documentclass[a4paper,11pt]{article}
\usepackage{jcappub}
\hypersetup{colorlinks=True,citecolor=blue}

\graphicspath{ {"figs/"} }


\begin{document}

\title{Spatial variations in the polarization power spectra of dust emission}
\author[a,b]{Aditya Rotti,}
\author[a]{Kevin Huffenberger,}
\author[c]{and Marc Kamionkowski}
\affiliation[a]{Department of Physics, Florida State University, Tallahassee, FL 32306, USA}
\affiliation[b]{University of Manchester}
\affiliation[c]{Johns Hopkins University}
%\date{\today}


\abstract{We compute the $E$- and $B$-mode polarization power spectra the 353 GHz sky map from Planck.  With an independent pipeline, we reproduce the global, high-latitude power law fit of $D^{EE,BB}_\ell =  A^{EE,BB}_{\ell=80} \left( {\ell/80}\right)^{2 + \alpha}$, with $\alpha \simeq -2.43$ and $A^{BB}_{\ell=80} \simeq 0.5 A^{EE}_{\ell=80}$.  We further break the sky into $11^\circ$ radius patches and power law parameters for each patch.  The distribution of parameters is broad enough to indiicate that there some significant spatial variation in the slope and ratio.  For certain locations (even $>30^\circ$ from the Galactic plane) we find patches that significantly deviate from the global mean, with significantly more $E$ power, more $B$ power, or a differing slope.  These are often associated with bright features that are visible directly in 353 GHz data, neutral hydrogen, or CO emission.
}

\maketitle

\section{Introduction}


\section{Methods}

To generate the scalar E/B maps from Stokes parameters Q/U, we use a  pipeline based on the methods of \cite{Ferte2013, Kim2010a} to correct for mask leakage by taking the transform of the map, apodization mask, and their derivative.  Our pipeline leaves leakage of power from primordial $E$ to $B$ is at a level comparable to a $B$-mode signal at $r=10^{-7}$, well below the target of current and planned experiments.


\section{Results}


\section{Discussion}


\section{Acknowledgments}



\end{document}
